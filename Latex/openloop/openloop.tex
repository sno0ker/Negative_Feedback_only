\documentclass[11pt]{article}
\usepackage[utf8]{inputenc}
\usepackage[T1]{fontenc}
\usepackage{graphicx}
\usepackage[portuguese]{babel}
\usepackage{geometry}
\usepackage{listings}
\usepackage{hyperref}
\usepackage{bookmark}
\usepackage{float}
\graphicspath{ {./img/} }
\begin{document}

\title{Malha aberta}
\maketitle

\section{Modelo de simulação}
\begin{figure}[!h]
\begin{center}
\includegraphics[width=12cm]{modelo.PNG}
\caption{Diagrama de blocos}
\label{fig1.1}
\end{center}
\end{figure}
\begin{figure}[!h]
\includegraphics[width=15cm]{params.png}
\caption{Inicialização de parâmetros e matrizes}
\label{fig1.2}
\end{figure}

\pagebreak
\section{Simulações}
Para melhor entendimento do sistema a controlar e como meio de podermos concluir quais as respostas do sistema a mudanças de parâmetros, serão efetuadas algumas simulações.
Inicialmente será testada a resposta do sistema a mudanças de valores das resistências hidráulicas e de seguida a mudanças do valor de área dos 3 tanques, estes que se encontram a 9$m^2$ inicialmente.
Note-se que quaisquer comparações referidas nos comentários de cada simulação, estarão a ser feitas em relação à figura \ref{fig2.1}.
\subsection{Rh1=1$\frac{m}{m^3/s}$, Rh2=1$\frac{m}{m^3/s}$, Rh3=1$\frac{m}{m^3/s}$}
\begin{figure}[!h]
\includegraphics[width=16cm]{initial.png}
\caption{Rh1=1$\frac{m}{m^3/s}$, Rh2=1$\frac{m}{m^3/s}$, Rh3=1$\frac{m}{m^3/s}$}
\label{fig2.1}
\end{figure}
Tendo todas as resistências hidráulicas como valor 1, e sabendo que qi1=5$m^3/s$ e qi2=2$m^3/s$, observa-se que em regime permanente h1 = 5m, h2 = 7m e h3 = 7m. h1 é o primeiro a entrar em regime permanente, seguido de h2, e  por último, h3.
\pagebreak
\subsection{Rh1$\frac{m}{m^3/s}$=2,Rh2$\frac{m}{m^3/s}$=1,Rh3=1$\frac{m}{m^3/s}$}
\begin{figure}[!h]
\includegraphics[width=16cm]{rh1.png}
\caption{Rh1=2$\frac{m}{m^3/s}$, Rh2=1$\frac{m}{m^3/s}$, Rh3=1$\frac{m}{m^3/s}$}
\label{fig2.2}
\end{figure}
Alterando apenas o valor de Rh1 para o dobro, verifica-se uma resposta mais lenta do sistema, de forma que todas as variáveis de saída demoram mais tempo a atingir o regime permanente. Verifica-se tembém que o valor em regime permanente h2 e h3 não sofre alteração, mas o valor de h1 passa a ser o dobro, o que faz sentido uma vez que o output de água do tanque 1 passa para metade, logo acumula.
\pagebreak
\subsection{Rh1=1$\frac{m}{m^3/s}$,Rh2=2$\frac{m}{m^3/s}$,Rh3=1$\frac{m}{m^3/s}$}
\begin{figure}[!h]
\includegraphics[width=16cm]{rh2.png}
\caption{Rh1=1$\frac{m}{m^3/s}$, Rh2=2$\frac{m}{m^3/s}$, Rh3=1$\frac{m}{m^3/s}$}
\label{fig2.3}
\end{figure}
Repondo o valor de Rh1 a 1 e aumentando agr Rh2 para o dobro, verifica-se que valor de h1 não sofre qualquer alteração em nenhum estado comparativamente com a simulação inicial(\ref{fig2.1}), isto é, não altera ser valor em regime permanente, nem o tempo que demora a atingi-lo. Quanto ao valor de h2 e regime permanente, como previsto, passa para o dobro, demorando também mais tempo para o atingir. O valor em regime permanente de h3 mantém-se ainda em 7m, demorando apenas mais tempo a ser atingido uma vez que o caudal de saída do tanque 2 é menor. 

\pagebreak
\subsection{Rh1=1$\frac{m}{m^3/s}$,Rh2=1$\frac{m}{m^3/s}$,Rh3=2$\frac{m}{m^3/s}$}
\begin{figure}[!h]
\includegraphics[width=16cm]{rh3.png}
\caption{Rh1=1$\frac{m}{m^3/s}$, Rh2=1$\frac{m}{m^3/s}$, Rh3=2$\frac{m}{m^3/s}$}
\label{fig2.4}
\end{figure}
Com Rh1=1$\frac{m}{m^3/s}$, Rh2=1$\frac{m}{m^3/s}$ e Rh3=2$\frac{m}{m^3/s}$ (dobro), verifica-se que h1 e h2 não sofrem quaisquer alterações em relação à simulação inicial (\ref{fig2.1}). Quanto a h3, como esperado, o seu valor em regime permanente passa para o dobro, visto que a quantidade de água que sai do tanque 3 passa para metade, e o tempo que demora atingir o regime permanente aumenta também visto que o valor que atinge é maior e tem o mesmo caudal à entrada.
\pagebreak
\subsection{Rh1=1$\frac{m}{m^3/s}$,Rh2=1$\frac{m}{m^3/s}$,Rh3=1$\frac{m}{m^3/s}$, Área dos 3 tanques=3$m^2$}
\begin{figure}[!h]
\includegraphics[width=16cm]{A3.png}
\caption{Rh1=1$\frac{m}{m^3/s}$, Rh2=1$\frac{m}{m^3/s}$, Rh3=1$\frac{m}{m^3/s}$, área dos tanques=3$m^2$}
\label{fig2.5}
\end{figure}
Alterando apenas o valor da área, verifica-se que não há qualquer alteração nos valores das variáveis em regime permanente, apenas muda o tempo que demora para que estas lá cheguem.
\section{Conclusão}
Com as simulações efetuadas, foi possível tirar algumas conclusões:
\begin{itemize}
\item Alterações nos valores das resistências hidráulicas provocam uma alteração na altura que a água consegue atingir no/nos tanque/tanques em questão, não alterando o valor em regime permanente das alturas dos restantes tanques;
\item Alterações nos valores das resistências hidráulicas provocam uma variação no tempo de resposta do sistema, fazendo com que o regime permanente seja atingido mais tarde ou mais cedo para todas as alturas dos tanques cuja resistência hidráulica foi alterada, e todos os tanques que se seguirem. Tanques que se encontrarem antes dos que a resistência hidráulica foi alterada não sofrem quaisquer alterações, provando que o sistema é sem interação.
\item Alterações nos valores das áreas dos tanques provocam alterações no tempo de resposta do sistema, embora não influenciem os valores em regime permanente.
\end{itemize}

\end{document}
